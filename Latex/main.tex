\documentclass[12pt]{article}
\usepackage[utf8]{inputenc}
\usepackage{listings}
\usepackage{amsfonts}

\title{Generazione automa caratteristico parsing bottom-up LR(0) }
\author{
        Sandri Fabrizio \\ \\        
        Dipartimento di Ingegneria e Scienza dell'Informazione\\
        Università di Trento
}

\date{\today}

\begin{document}
\maketitle

\section{Introduzione}
In questo report si vuole analizzare ed implementare la procedura di generazione di un automa caratteristico per il parsing bottom-up di tipo LR(0). Fra i vari tipi di parsing visti a lezione il parsing LR(0) è un tipo di parsing per cui si legge l'input da sinistra a destra e si produce una derivazione di tipo rightmost a partire dalla grammatica data in input.

\paragraph{Il problema:} 
il primo step del parsing bottom-up è quello di generare un automa caratteristico a partire dalla grammatica letta in input per poi andare a formare una tabella di parsing. In questo report ci focalizzeremo solamente sulla parte di generazione dell'automa caratteristico.

\section{Definizioni}
L'automa caratteristico è formato da un insieme di stati interconnessi da una funzione di transizione $\tau$ definita su coppie di stati. Se diciamo S gli stati dell'automa e V il vocabolario della grammatica considerata allora definiamo la funzione di transizione
$$
\tau \colon (S, V) \to S
$$
Ogni stato contiene degli LR(0)-items : alcuni di questi item faranno parte del kernel, mentre altri fanno parte della closure del kernel.\\

La tecnica di costruzione dell'automa caratteristico è incrementale: andiamo a popolare un set di stati definendo mano a mano la funzione di transizione, fino a saturazione.


\section{Input e Output}

\subsection{Input}

\subsection{Output}

\section{Strutture dati}

\section{Testing}





\end{document}